\documentclass{vivid_layout}
\usepackage{verbatim}
% \makeprint is for printing and trimming on paper, w/crop marks.
% \makeprint

%% Required to build cover page
\title{上海交通大学 ACM班}{\fontsize{37.5pt}{15pt}\selectfont 学长去哪儿}
\date{\color{white} \today{} \textbullet{} Version 1}
\cover{img/cover}

%% Required to build "Meet the Author"
\author{Zhen Wei}{}

%% Required image for "About VividCortex"
\aboutvc{img/presenter}


\begin{document}
\maketitle		% Build the cover page



\tableofcontents	% Build the table of contents


\section{Introduction}
\addtocounter{section}{1}

小编的话:

仅仅是第一版草稿。很抱歉,一直到现在才出了草稿,会在不断改进的。希望能帮到大家。

越来越感觉ACM班真的是一个很特殊的班集体,仅仅是在群里发了一次链接就有这么多学长学姐分享经验,真的很感谢各位学长学姐的支持,包括那些没有透露姓名的学长学姐。希望ACM班的学弟学妹们能受益于这些宝贵的经验,都有好的发展。也想借这个机会谢谢俞老师,把我们聚在一起,也正因为俞老师对每个人的关心,ACM班才是一个如此有凝聚力的集体。

\section{About GPA}
\addtocounter{section}{1}
\setcounter{subsection}{0}

\subsection{GPA到底有多重要?}
GPA到底有多重要?GPA高不成低不就还要不要谈GPA的事?

\begin{itemize}
\item  {\name 03 刘渊杰}  \quad 取决于要不要出国读书 读书的话gpa还是很重要的 不然好学校会质疑你 另外也有一些看重gpa的公司
\item  {\name 02 杨林骥}  \quad 除了申请出国,别的不太重要 但是打好基础很重要
\item  {\name 11 杨宽}  \quad 我的感觉是除了Top 3或者说Top 10\%以外都没用,谈不谈GPA以自身实力为主,如果和CV上其他方面的成绩不匹配还是别谈了吧,或者笼统地提一下。
\item  {\name 09 朱虹宇}  \quad 很难说,取决于你的目标。好的gpa会给人一个好的印象,但也不是全部。如果你有能力把一个对自己有好处的事情做好,为什么不呢。 
\item  {\name 07 刘斌}  \quad 平时学习:非常重要,体现扎实基本功。但不要过度投入或者患得患失。申请:这个随意,看具体老师的偏好。
\end{itemize}

\subsection{如何提高GPA}

\begin{itemize}
\item {\name 匿名学长学姐}  \quad 考前刷题。认真对待每次作业、考试。
\item  {\name 匿名学长学姐}  \quad 多和老师同学交流讨论,多阅读,多练习,积极反思。
\item  {\name 匿名学长学姐}  \quad 熟知期末如何评分,通识政治和老师搞好关系,专业课认真上。
\item  {\name 09 李诗剑}  \quad 除了努力,其他的方法长期上来看最后都会害你。不要图一时的便利
\item  {\name 11 杨宽}  \quad 没意义。
\item  {\name 09 朱虹宇}  \quad 分课程具体看吧。个人感觉掌握一门课的思维方式最重要,如果你的思维方式对了,再认真一些,gpa不会低到哪里去的。
\item  {\name 07 刘斌}  \quad 不要懈怠打游戏,也不要蛮力搞学习,do it in a smart way。时间精力是有限的。
    和老师多沟通,避免备考方向偏了,或者准备不充分,从而在分数上吃亏
\end{itemize}


\subsection{课业方面如何自学?}
\begin{itemize}
\item {\name 匿名学长学姐}  \quad 个人认为有两种自学,一种是有目的性的,往往是碰到一个问题不会做,然后去补。这种比较有动力。还有一种是扩展性的,像学一些奇怪的数学,这就是贵在坚持了。
\item {\name 12 陈思奇} \quad 自学是建立在自己的兴趣基础上的,有了兴趣就会自主探索,就会成为自学的过程。
\item {\name 12 陈愉}  \quad  课程的话看考的难不难,难得话看教材,不难就看ppt。自己喜欢的方向或课程试图去深入的了解一下。要了解一个知识最快的方法我认为是对wiki或者paper进行dfs,当然基础要扎实的话还是找教材看吧,就是时间比较长效率不是很高,可能知识方法也不是非常的新。
\item  {\name 05 刘畅}  \quad 提高阅读效率的唯一办法就是多读多思考。边读边思考。
\item  {\name 09 李诗剑}  \quad 我觉得自学最重要的就是不要装糊涂,没看明白的一定要弄明白,不能马马虎虎的就过去了。
\item  {\name 09 朱虹宇}  \quad 如果你有足够的motivation自己都能找到答案。技术上你可以去找一些online course,有条件的话实验室里面积极一些
\end{itemize}

%%%%%%%%%%%%%%%%%%%%%%%%%%%%%%%%%%
\section{About Research and Paper}
\addtocounter{section}{1}
\setcounter{subsection}{0}

\subsection{科研方面如何自学?}
\begin{itemize}
\item {\name 09 刘正阳}  \quad  先找一片综述性的文章进行阅读,再根据引用进行深入学习(我自己也还没有做得很好~)
\item  {\name 05 刘畅}  \quad 提高阅读效率的唯一办法就是多读多思考。边读边思考。
\item  {\name 10 严欣辰}  \quad Research最好有个人带一开始
\item  {\name 03 刘渊杰}  \quad 做项目和实习是最好的方式
\item  {\name 11 杨宽}  \quad 首先,需要一本合适的教材,然后,需要一个熟悉这个方向的老师。老师会告诉你如何入门,如何逐渐深入
\item  {\name 07 刘斌}  \quad 最好别自己瞎找,多和相关方向的老师和师兄师姐沟通,自己找材料容易迷失方向或者跳步超纲。
\item  {\name 12 谢其哲}  \quad 自己入门比较难。对于大部分本科同学来说,好的导师和好的学生导师很重要。taste好的导师能发现领域里重要的问题,给一个很好的topic,决定做什么。但导师没时间做细节的指导,靠谱的学生导师能告诉你怎么做,如果你发现idea不work,他们能用经验告诉你应该从哪些方向分析。做什么和怎么做都很重要,可能本科期间主要需要打基础,学习怎么做,保证自己不会因为解决问题能力不够而错失机会。在PhD期间需要自己关注做什么,方向不对付出精力再多也徒劳。所以我一直很感谢我的导师俞凯老师和带我的学长孙锴。据我所知ACM班的很多学长学姐很靠谱。
\end{itemize}

\subsection{publication的历程以及其在申请有怎样的影响}
\begin{itemize}
\item  {\name12 尹茂帆} \quad 导师开脑洞,勤恳做实验,仔细写论文,好好画插图。交Interspeech,拒。reviewer主观情绪重,为喷而喷。逢Cornell交流,再投ASRU。新加坡实习期间收二拒。悲伤之余还得继续carry on手头的新研究,在实验结果很不理想的情况下赶着9月底的deadline,准备着8月底的托福。托福一战顺利,后来才知道原来ASRU还有一篇我二作的文章(一作是老师)中了,一作把邮箱填错了所以我没收到邮件,心情平复了些许。最后赶在deadline前夜把结果跑正常了,提交到语音顶会ICASSP。然后把重心放在申请准备。最后申请时CV上有两篇submitted的paper(一篇是语音顶会一作),一篇搁置的paper in progress(其实已经不太可能再投),三篇录取的paper(分别是二、三、四作)。12月15日deadline过后的22日得到消息:我实习的paper录了。赶紧联系学校更新CV,好几所都愿意。
\item  {\name 09 刘正阳}  \quad 被拒了两次,降档投中了。没申请过。
\item  {\name 04 李博}  \quad 本科没有论文。读博以后才有的。有论文对申请帮助蛮大的
\item  {\name 05 刘畅}  \quad 因为我在出国前发的顶级会在美国发的人并不多,所以帮助并不是很大。但是如果在美国的主流会发paper帮助是很大的。对于欧洲的情况不了解
\item  {\name 02 杨林骥}  \quad 写出来几篇总是有很大的正面影响的。现在申请人大多数都有几篇
\item  {\name 09 朱虹宇}  \quad 个人并没有一作publication。基本上我都是跟班的状态
\item  {\name 07 刘斌}  \quad 本科时投了两次均没中,只中了一个很水的workshop。申请的时候电话面试,老师都并没有具体问到paper的事情,只是问之前在实验室和实习都具体做哪些事情。(可能跟我并没有申请领域大牛老师有关系)
\end{itemize}

\subsection{如何读paper?}
\begin{itemize}
\item  {\name 12 陈愉}  \quad  先找idea再看细节,如果方法了解只是计算的话就不用看细节了,如果有一个好的methodology的话细节就很重要了。
\item  {\name 09 刘正阳}  \quad 认真读,每个字都要看
\item  {\name 12 田博雨}  \quad 如果一篇文章的背景你是清楚的,那么你可以跳过没用的部分(abstract, introduction, background);如果背景不是十分清楚,并且paper中没有完整详细的介绍,那么就需要找相关背景最初的几篇以及对这篇paper影响最大的几篇先大致读一下。只是粗读一篇paper的话,证明可以不求甚解。
\item  {\name 04 李博}  \quad 看你要怎么用论文了,如果只是了解性质的,看看abstract,大概扫一下文章就好了。这种情况目的是给自己建立脑内的索引,万一需要的时候可以找出来再看。
\item  {\name 05 刘畅}  \quad 多读。我发现写paper,或者写notes(用英文)对于读paper有帮助。因人而异
\item  {\name 10 严欣辰}  \quad 多看几遍,好的文章值得过一段时间回来看看
\item  {\name 02 杨林骥}  \quad 读个大概思路 从大局出发
\item  {\name 09 朱虹宇}  \quad 先搞清楚问题是什么,再搞清楚这个paper做了哪些contribution。
\item  {\name 07 刘斌}  \quad 对重要的或者强相关的paper一定要反复读,而且一定要求甚解,否则后面具体自己研究东西写code的时候都会把欠下的找回来。而不重要的paper主要抓idea和challenge就可以了。
\item 假想你在审稿,想把这篇文章拒掉又不想得罪人,需要找一个非常强的理由拒
\end{itemize}

\subsection{在申请时候不同情况的paper申请效力上差多少?}
比如一作or非一作 A会orB会 accepted/submitted
\begin{itemize}
\item  {\name 04 李博}  \quad 一作比非一作强很多。如果非一作的话,一定要在personal statement里说明哪些工作是自己做的。
\item  {\name 05 刘畅}  \quad 我自己在admission committee review材料的时候不太看是不是一作。但是我们会仔细找证据证明一个学生对于一个paper的contribution有多少。所以在PS里展现对于自己paper非常深刻的理解是非常有帮助的(不过大部分申请者都做不到)
\item  {\name 10 严欣辰}  \quad Paper越solid越好
\item  {\name 12 陈愉}  \quad theory的会都是平行作者。。。。
\item  {\name 02 杨林骥}  \quad 理论不分顺序 会议好坏差很多 老美只看soda focs stoc
\item  {\name 11 杨宽}  \quad 我感觉非accepted的paper都是废纸,今年有本科生发了STOC还被四大拒光的,简直难以理解。
\item  {\name 09 朱虹宇}  \quad 分方向吧。一作肯定是很有加分的。
\item 不计submitted,好paper加一分不好paper减一分算总分,然后再看是否一作
\end{itemize}

%%%%%%%%%%%%%%%%%%%%%%%%%%%%%%%%%%
\section{About Lab}
\addtocounter{section}{1}
\setcounter{subsection}{0}
\subsection{在实验室需要参与哪些工作?码代码在宿舍码不是也行吗?}

\begin{itemize}
\item {\name 匿名学长学姐}  \quad 主要是和别人进行交流。而且实验室也有更好的设备。
\item {\name 06 吴天翔} \quad 记得实验室老师提到,在实验室看论文和写程序也是一种氛围,气氛及心境对工作效率是很相关的。在实验室还能就问题向学长学姐请教。实验室的工作主要包括研究课题相关前沿,拍脑袋想方法,实现及测试,还有维护实验室卫生。
\item  {\name 12 陈愉}  \quad 反正我没去过实验室。。。。我感觉这和为什么自修要去图书馆是一回事
\item  {\name 09 刘正阳}  \quad 和学长学姐进行交流最重要了
\item  {\name 05 刘畅}  \quad 我自己在学校不参与工作(感觉不太好)。不过自己lead的project需要跟很多人开会,保证所有人的进度。交流的意义在于避免码没必要的代码,并且保证自己的代码可以被别人用。
\item  {\name 10 严欣辰}  \quad 主要是和人交流idea
\item  {\name 09 朱虹宇}  \quad 基本上是你在一个research project中应该扮演的角色。不仅仅是对自己的部分要熟悉,对队友的工作在讨论时也要有所了解。码代码这种,如果是一个team work,那么在实验室可能大家讨论会方便一些。另外我当时常去实验室的一个很重要的原因是我有融入实验室的圈子,据我所知一些导师也希望acm班的同学能够更融入实验室
\item  {\name 07 刘斌}  \quad 实验室的好处在于你可以和志同道合的人在一起,讨论方便,而且互相之间也有最新的信息。teamwork is always better。weekly meeting presentations和paper reading group也很有帮助。
\end{itemize}

\subsection{介绍下自己曾经待的实验室和自己的导师吧}
\begin{itemize}
\item  {\name 12 陈愉}  \quad  aims,邓小铁老师的实验室,陆品燕老师离开以后貌似就邓老师一个做理论的了吧(张志华老师是理论机器学习)。主要方向是博弈论相关(解决一些有趣的问题,上手快也不太需要写代码,但是难以出成果)
\item  {\name 09 刘正阳}  \quad aims实验室,邓小铁老师。氛围很融洽,老师不会很push,关键在于自觉。
\item  {\name 05 刘畅}  \quad 我在APEX,yyu跟Haofen手下工作。master阶段跟东南大学的Guilin合作很多。最后一年在MSRA的Lidong组工作。Lidong对我的申请帮助非常大。他发个邮件给Mike(我phd导师),然后我就拿到offer了。
\item  {\name 11 杨宽}  \quad 我本科时的导师是陆品燕教授,但由于陆老师全职加入上海财经大学了,我不确定这个选项现在还存在。对理论,尤其是算法与复杂度方向有兴趣的学生建议联系Dominik或者张驰豪。
\item  {\name 09 朱虹宇}  \quad EPCC,导师是过敏意老师。他自己level比较高,平时不太抓实验室的工作。你会参与他手下的一个小导师或者博士生的组。这个实验室因为做的系统方向,和业界联系会比较多,这是一大优势。劣势的话则是不太容易出文章,但你会学到一些硬功夫。
\item  {\name 07 刘斌}  \quad 俞老师APEX,俞老师,薛贵荣老师。启蒙老师,感恩一辈子。
\end{itemize}

%%%%%%%%%%%%%%%%%%%%%%%%%%%%%%%%%%%
\section{About Internship}
\addtocounter{section}{1}
\setcounter{subsection}{0}

\subsection{如何更好的利用实习机会?}
\begin{itemize}
\item {\name 匿名学长学姐}  \quad 实习的时候正好要忙申请,其实挺冲突的。主要还是搞好手头工作,多多学习。
\item {\name 12 陈思奇} \quad 多参加会议,多和同时期的实习生见面,和研究员交流。不要把自己局限于自己在的组内。
\item  {\name 09 刘正阳}  \quad 研究一个问题,把它做完整。
\item  {\name 05 刘畅}  \quad 努力工作。发挥自己的特长,让周围的人,尤其是你的大老板,觉得你(1)非常好相处;(2)很强。
\item  {\name 09 李诗剑}  \quad 实习头两个月大概都要融入一段时间。不要上来就干活,先弄明白项目的目的和影响力,再找着力点。不要钻牛角尖。
\item  {\name 10 严欣辰}  \quad 进Paper-oriented 的team
\item  {\name 02 杨林骥}  \quad 多动手 还有多和人交流 不要太孤立自己
\item  {\name 11 杨宽}  \quad 尽量多认识不同的老师,和不同的老师多交流。
\item  {\name 09 朱虹宇}  \quad 应该做什么,实习的导师都会给你布置的。重要的是观念的转变。你不是去打工不是去帮忙,你是去真正take on一个自己的项目,可能这个项目一开始不是那么大,但是要把自己放到一个具有主导权的位置上。另外就是一定不要对队友的工作一无所知。这些东西你在实验室甚至做course project也是一样的。这样才能把事情做好。
\item  {\name 07 刘斌}  \quad 把实习的事情做好,多和公司的同事交流。
\item  {\name 12 谢其哲}  \quad 做好手头上的事情后主动找事情做,多跟资深大牛交流。
\end{itemize}

\subsection{实习和实验室的区别是什么?}
实习期间的research感觉和在交大实验室时的research有什么不同(不精确到research topic的,只是在high level层面的)?别的方面有不同么?

\begin{itemize}
\item {\name 匿名学长学姐}  \quad 微软有些组很资源很充足,想要设备可以很容易搞定,但是在交大可能科研经费不会像微软那么多。而且微软实习的时候大家各个组的人都坐在一块,可以遇到很多其他的学校的人。和他们交流收获很多。
\item  {\name 12 陈愉}  \quad 实习的时候跟的是AP,所以我可以跟着他一步步做,在学校的话邓老师基本只跟你讲high level的事情。最多讨论的时候给你出出主意。
\item  {\name 05 刘畅}  \quad 在公司的项目会跟产品相关,而且关系密切。如果到国外的大学的lab实习,跟国内实验室最大的区别在于topic不一样。在国外实验室选的topic会跟学术和工业前沿贴得非常近。
\item  {\name 09 李诗剑}  \quad 实验室注重的影响力比较长期(当然取决于你的实验室)。实习的关注点比较接近当前热点和行业前景。
\item  {\name 10 严欣辰}  \quad 得到的帮助更多吧,实习比较focus
\item  {\name 11 杨宽}  \quad 我本科导师,也是实习的导师陆品燕教授当时还是微软亚洲研究院的研究员,所以对我来说没区别。大概是以前觉得见他一面不容易要走好远,后来可以天天见而且也不觉得微软离交大远了。
\item  {\name 09 朱虹宇}  \quad 我觉得这个问题问的不是很好。最重要的是你做的事情是不是你喜欢做的,以及是不是值得一做的,其他的相对比较浮云吧。喜欢的事情和有价值的事情在哪做都没差,最大的区别可能也就是实习钱多点。
\item  {\name 07 刘斌}  \quad 实习的时候可以得到不同的资源(比如更多data),可以得到更多人的指点和合作。而且实习的时候全天候做事,没有上课开会等事情打扰,可以更专注。
\end{itemize}

\subsection{实习时如何抱大腿and勾搭老板?}
实习除了拼死干活以外如何和老板social,让他给你强推?
\begin{itemize}
\item {\name 匿名学长学姐}  \quad 敢于交流就好,大神都很不错的。(只要他们有时间)
\item  {\name 09 刘正阳}  \quad 这些问题都不用去想,自己认真做了,对得起自己就好
\item  {\name 05 刘畅}  \quad 努力干活是最重要的,要做到非常非常的focus。另外要增加自己的存在感,开会的时候多发言,做到发言有point很重要。
\item  {\name 09 李诗剑}  \quad 1.你干的一定要出色 2.适当提高自己的visibility,比如定期做个报告。
\item  {\name 10 严欣辰}  \quad 这个和与人交往差不多吧,顺其自然
\item  {\name 02 杨林骥}  \quad 必须要让他欣赏你
\item  {\name 11 杨宽}  \quad 多交流,不一定要拼死干活,让导师知道你做了什么,展示出你的能力和想法更重要。
\item  {\name 09 朱虹宇}  \quad 表现出你的优势。如果你有一些吸引人的特质,大家都会看在眼里,之后的一切不过是刚刚好而已。social的话,先从一起吃饭扯皮开始吧。
\item  {\name 07 刘斌}  \quad 做事靠谱,而且有效沟通就可以了。老板充分了解你做的事,自然心里就有数了。
\item {\name 12 谢其哲}  \quad “A lot of people want a shortcut. I find the best shortcut is the long way, which is basically two words: work hard.” ——Randy Pausch
\end{itemize}

\subsection{国外教授与中国教授有什么不同}
在进行国外的研究生活中,有什么需要注意的? (比如经常听人说国外的教授强调积极主动,要多跟国外教授展现自己并说出自己想 要什么之类的)
\begin{itemize}
\item {\name 12 陈思奇}\quad 在研究活动中,我认为你与教授应当是合作关系,而不是师生关系。所以你不能等着教授给你布置任务,告诉你该怎么做,下一步干嘛。要自己主动钻研课题,寻找思路。
\item  {\name 05 刘畅}  \quad 在美国这边如果你没有办法让教授感觉到你的存在感,那基本只是起到帮教授消耗funding的作用。没有人会对你impressive,更不可能对你有非常有效的推荐。
\item  {\name 02 杨林骥}  \quad 积极主动 不要怕丢脸 多问傻问题
\item  {\name 09 朱虹宇}  \quad 国外教授分人看吧,不好笼统的说。积极主动在哪里都是一样的,这个还是在于说你是否把一件事情当做是自己在主导而不是仅仅去帮忙。单就我的感觉来说,国外环境会更加宽松,你自己对自己的安排和规划也没那么多拘束
\item  {\name 07 刘斌}  \quad 具体看人,并没感觉有什么国内国外之分
\item {\name 匿名学长学姐}  \quad 人没啥不同,但制度不同,要搞清楚规则
\end{itemize}

\subsection{推荐到什么地方实习?如何去联系实习的老师呢?}
\begin{itemize}
\item {\name 匿名学长学姐}  \quad 微软还挺好的。
\item {\name 12 陈思奇}\quad 直接发邮件,问他可不可以参与实习,告诉他你为什么想实习,你有什么能力,你能为他做什么。如果你有熟人(比如参加比赛,认识了大牛,他说能够帮你推荐找实习)那是很好的,因为你知道自己将在实习时候做什么,以及自己是不是对这个实习内容感兴趣。
\item  {\name 05 刘畅}  \quad MSR Redmond, top 10 university lab. 通过学长学姐联系应该是最有效的。
\item  {\name 10 严欣辰}  \quad 自己联系,看兴趣和老板最近3年的publication
\item  {\name 11 杨宽}  \quad 海外高校
\item  {\name 09 朱虹宇}  \quad 牛人多的地方。如果你要自己联系老师的话,一定要让俞老师跟对方沟通,这样可以避免一些误会。
\item  {\name 07 刘斌}  \quad 研究性质的机构(比如MSRA)俞老师很给力的,师兄师姐的人脉也能帮上忙。 
\item {\name 匿名学长学姐}  \quad 大公司,名校,大牛,口碑。联系方法发email
\end{itemize}

%%%%%%%%%%%%%%%%%%%%%%%%%%%%%%%%%%%
\section{About English}
\addtocounter{section}{1}
\setcounter{subsection}{0}

\subsection{需要参加托福和雅思和GRE的培训班吗?效果如何?}
\begin{itemize}
\item  {\name 匿名学长学姐}  \quad 不需要,效果不是很好。
\item {\name 12 陈思奇} \quad 可以
\item 个人认为不需要。我自己感觉最有用的部分是GRE的作文,但是其实相关的资料网上也能找到。关键是要下功夫准备,如果你觉得上课对自己也是一种推动作用的话也可以上课。
\item  {\name 12 陈愉}  \quad 感觉新东方托福强化班主要告诉你如何提高到90分,再高还是靠自己。
\item  {\name 12 田博雨}  \quad 新东方会交给你考试考些啥,怎么答,怎么准备,如果你觉得你需要有人就这些方面指导你,那么参加课程是不错的选择。然而这些课程并不会帮你准备,准备考试要靠自己。参加课程的另一个好处就是,你会认识一些一起准备的小伙伴,大家互相督促,当然,和室友一起搞也是很好的选择。
\item  {\name 05 刘畅}  \quad 我参加过GRE的培训班。完全没用
\item  {\name 09 李诗剑}  \quad 取决于个人英语能力
\item  {\name 09 朱虹宇}  \quad 看你自己的需要了。我自己新东方上托福的经验,写作提高了3分,口语还是不变。
\item  {\name 07 刘斌}  \quad (1)培训班主要是可以加速入门,师傅领进门,修行在个人。
  (2)可以认识学友,可以互通有无。
    (3)等于是花钱买个鞭子,总上课的话可以督促备考,减少自控力差的不良反应。
\end{itemize}

\subsection{ 什么时候报名考试比较合理?分配多长时间准备考试呢?}
\begin{itemize}
\item  {\name 匿名学长学姐}  \quad 越早越好!
\item {\name 12 陈思奇}\quad 大一/大二/大三上
\item 尽量早些吧。不过注意TOEFL的有效期只有两年,应该大三再考。准备时间三个月到半年这样就可以。个人觉得最主要的是背单词,还有TOEFL口语。
\item  {\name 12 陈愉}  \quad 托福平时有空的时候做做tpo,gre的话短时间突击至少150是没问题的。在学有余力的前提下当然越早越好
\item  {\name 12 田博雨}  \quad 越早越好!准备的时间要看自己的英语水平而定,以达到最低标准为目标,不要浪费更多的时间。
\item  {\name 05 刘畅}  \quad 努力准备吧,如果实习做得好其实不太需要管GRE跟TOEFL。在admission committee里确实会有老师会concern一个toefl太低的考生的英语能力。但是只要你的recommender在推荐信里说在美国实习期间与人交流没有问题就好了
\item  {\name 10 严欣辰}  \quad 看个人,可以先去考一次
\item  {\name 09 朱虹宇}  \quad 有空的时候就应该提上议事日程了(背单词啥的)。一般快的大二准备,慢点的话大三准备,我们那时大一考g的人也是有的。
\item  {\name 07 刘斌}  \quad 因人而异,如果基础差就早准备多准备,基础好就没必要占用太多专业课时间了。
\item  {\name 匿名学长学姐}  \quad 报名太晚可能会发现上海报满了,甚至耽误申请。分配一个月准备
\end{itemize}

\subsection{托福和GRE到底会不会卡分呢?}
\begin{itemize}
\item  {\name 匿名学长学姐}  \quad 会
\item {\name 匿名学长学姐}  \quad  toefl 100很多学校都看,而且真的会被argue。GRE最好325,没到偶尔会被argue。
\item  {\name 12 田博雨}  \quad 会,但是不至于卡死,很多学校的标准是会松动的。
\item  {\name 05 刘畅}  \quad 如果committee里有人知道你,那么就不会卡。Cornell会卡.
\item  {\name 10 严欣辰}  \quad 别太低就行
\item  {\name 11 杨宽}  \quad 不知道。我猜不会,英语成绩应该不会成为决定性的因素。
\item  {\name 09 朱虹宇}  \quad 分学校吧,有的是会的(好像康奈尔是比较著名的卡分学校)。一般gre很少卡,对托福有明确要求的会多一些。尽力做好就好。
\item  {\name 07 刘斌}  \quad 有些学校会卡,小秘会事先筛掉一部分申请。
\item  {\name 12 谢其哲}  \quad 申Master最好英语比较高。PhD的话TOEFL会卡,GRE不会。我浪费了太长的时间在刷GRE上。我觉得做每件事情前都应该想想这件事值不值得花这么多时间,人的精力很有限,不要尝试把每件事情都做好,老师们也不会看你的GRE,你可能会发现很多厉害的PhD不会去刷高GRE,因为他们都把时间拿去做科研了。
\end{itemize}


%%%%%%%%%%%%%%%%%%%%%%%%%%%%%%%%%%%
\section{About Application}
\addtocounter{section}{1}
\setcounter{subsection}{0}

\subsection{如何防止和战胜申请拖延症?如何合理安排时间呢?}
\begin{itemize}
\item  {\name 05 刘畅}  \quad 提前一年开始联系国外的实习。可以考虑出国之前把GRE跟toefl考掉,这样可以在实习期间focus在research上。
\item  {\name 02 杨林骥}  \quad 有使命感
\item  {\name 12 赵卓越}申请拖延症未必不好,只要最终不要来不及就行了。
\item  {\name 09 朱虹宇}  \quad 列计划,这周要做什么下周要做什么,然后就是做任务的概念了。
\item  {\name 07 刘斌}  \quad 抱团组团,互相勉励,可以减少拖延。
    个人反对刻意做日程安排,如果心里有数自己要做什么,就自然按部就班做。心里没想着要做什么,心里懈怠了,再怎么安排时间表也不会按计划完成。态度最重要。
\item  {\name 12 谢其哲}  \quad 有一种拖延很好:一直拖不重要的事情,把重要的事情做了。有一种拖延是因为不知道自己想要什么,也就做什么都没有动力。我在大三大四的时候在各种书中看了不少不同的人生,知道了自己想成为什么样的人,发现想做的太多了时间太少了,就不那么拖了。
\end{itemize}

\subsection{ 申请学校是不是越多越好?}
当然可能精力有限主申几所,其他的就直接照搬材料,如果这样的话不是越多越好吗?
\begin{itemize}
\item   {\name 匿名学长学姐}  \quad 不是,申不动了。。。我觉得只申,如果只录了这一所,肯定会去的学校。
\item {\name 12 陈思奇}\quad 申请费用很贵的。
\item  {\name 12 田博雨}  \quad 1、同一个层次的学校申请的数量和被录取的概率无关。2、申请是要花钱的。3、考虑一下别人,大家都把所有学校申一遍的话所有的offer就都给那么几个人了吧
\item  {\name 05 刘畅}  \quad 你申请学校越多,对你的recommender的压力越大,因为他们每个学校需要花他们五分钟的时间来提交申请。所以越早的把你的信息提交给你的recommender越好。
\item  {\name 10 严欣辰}  \quad 每一所都得认真对待,有精力多申请几所
\item  {\name 11 杨宽}  \quad 不一定,最好层次分明
\item  {\name 09 朱虹宇}  \quad 一般没有啥坏处,只是多花点钱,再多花点精力填几张表。
\item  {\name 07 刘斌}  \quad 保底校两三个就可以,其他的最好是申只要中了就愿意去的学校,否则中了也是空欢喜没啥用。
\end{itemize}

\subsection{申请时候要套磁么?}
套多少老师比较恰当呢?
\begin{itemize}
\item   {\name 匿名学长学姐}  \quad 需要。4~5个吧,最好要有引荐人,然后发邮件的时候refer一下引荐人。不然就不完全算陶瓷,大概是导师翻申请材料的时候会优先翻邮箱。
\item   {\name 匿名学长学姐}  \quad  可以发一些邮件,很多老师也会回复。如果你也读过对方写的论文,那可以深入交流,否则一般对方也只会说一些客套话。不过即使如此让老师留个印象也是不错的。
\item  {\name 12 陈愉}  \quad 我的陶瓷基本都是套路碰到套路,没什么用。建议套保底校和跟你推荐人比较熟的(有一定connection)
\item  {\name 05 刘畅}  \quad 大部分顶级学校不要套词。部分学校可以。我完全没套过,所以不知道。
\item  {\name 10 严欣辰}  \quad 要的很有帮助如果自己有短板的话
\item  {\name 11 杨宽}  \quad 一所学校不要超过两个人,套多了万一老师之间有交流就傻了。
\item  {\name 09 朱虹宇}  \quad 我自己没有套辞的经验,只是套辞之前你要对这个导师的工作有所了解,最好读过他的文章,否则意义不大。
\item  {\name 07 刘斌}  \quad 如果是导师决定,陶瓷很有用,多多益善,看个人精力。老师每天收到大量申请,谁更主动就可以赢得更多机会,只要不让他觉得烦。如果是committee制,除非可以和特别想跟的老师有联系(认识,或者导师认识),否则没有什么效果,把申请材料准备好一些就好。
\end{itemize}

\subsection{对于不是我Major的导师,怎样让他知道我们牛}
\begin{itemize}
\item   {\name 匿名学长学姐}  \quad  需要引荐人。
\item  {\name 05 刘畅}  \quad 首先要想清楚为什么要让其它导师觉得你牛。如果你不跟别的导师做工作,那么他们的推荐信基本没有用。所以这个问题是,如果能在跟自己的导师一起合作发paper的同时,还有时间跟别人合作发paper。这个就需要一个非常好的时间管理。但是,要确保跟自己的主导师的work要完成的非常perfect。
\item  {\name 09 李诗剑}  \quad 争取到聊一次,你到底牛不牛教授问你几个问题一下就知道了
\item  {\name 02 杨林骥}  \quad 自述里面 提自己对他领域的看法
\item  {\name 11 杨宽}  \quad 讨论学术问题
\item  {\name 09 朱虹宇}  \quad 不用想太多,把自己最真实的一面展现出来就好。如果你展现出来的不是你自己,就算给你offer你将来也未必坚持的下来。
\item  {\name 07 刘斌}  \quad 硬性条件,做过的项目,具体描述体会和解决到的困难
\item  {\name 12 谢其哲}  \quad 自己的水平很重要,好学生很多,所以他们倾向于找在本领域有成果的。如果他认识你的推荐人,而且你的推荐人很看好你,这也能说明你的水平高。
\end{itemize}

\subsection{同一个班内的申请听说存在竞争关系,这是否属实? }
\begin{itemize}
\item   {\name 匿名学长学姐}  \quad 不存在。外面那么广阔的天空,一个班就那么点人,怎么竞争?除非你们班三个人都申同一个老师。
\item  {\name 12 陈愉}  \quad 我觉得同一个方向可能有,但是没那么严重
\item 属实
\item  {\name 05 刘畅}  \quad 不存在
\item  {\name 09 李诗剑}  \quad 我没遇到过,不过听说有
\item  {\name 10 严欣辰}  \quad 是的
\item  {\name 09 朱虹宇}  \quad 看对方学校的处理申请的流程,不是很重要。该是你的就是你的,不该是你的你也求不到。
\item  {\name 07 刘斌}  \quad master的话可能有(项目要保证diversity),phd应该没有。学校招生数小的话可能有(比如普林,哈佛等)
\item   {\name 匿名学长学姐}  \quad 有控制diversity的说法,除此之外就没听说过了
\item  {\name 12 谢其哲}  \quad PhD不会
\end{itemize}

\subsection{如何获取推荐信?}
强推的获取之道有什么?对于国外老师(尤其不是大四实习所跟的导师),又如何获得呢(有其他途径么)?
\begin{itemize}
\item  {\name 05 刘畅}  \quad 只需要一封非常非常强的推荐信申请phd就足够了。推荐信获取的唯一途径就是跟他们非常有效的collaborate。
\item  {\name 10 严欣辰}  \quad 找最熟悉自己的人
\item  {\name 11 杨宽}  \quad 开会时多认识人,出国交流时多认识人,做好connection,多讨论一些学术问题。对于不熟的老师还是算了吧。
\item  {\name 09 朱虹宇}  \quad 找了解你的老师,找那些你跟他合作愉快的导师。要想获得强推,先跟推你的老板有比较深入的交流(比如课程中表现积极,或者research project表现出色)。
\item  {\name 07 刘斌}  \quad 做事靠谱,多沟通(自己做了东西老师不知道,就没法acknowledge你)
\item {\name 匿名学长学姐}  \quad 两个途径。一是在导师面前表现。二是在导师非常信任的手下面前表现。
\end{itemize}

\subsection{申请美帝以外的学校有什么经验和建议吗?}
\begin{itemize}
\item  {\name 09 李诗剑}  \quad 英国这边更看硬实力。找好的导师对的方向比学校重要
\item  {\name 11 杨宽}  \quad 其实差不多,可能需要考雅思。
\item  {\name 09 朱虹宇}  \quad 我自己是在加拿大,反正申请起来感觉跟美国差不多。加拿大学校也认托福反正
\item  {\name 07 刘斌}  \quad 德国的硕士花费很低, 有些比如TUM和Frieburg不用学德语
\end{itemize}

%%%%%%%%%%%%%%%%%%%%%%%%%%%%%%%%%%%
\section{About Life}
\addtocounter{section}{1}
\setcounter{subsection}{0}

\subsection{有必要一定要在学习之外找到一件喜欢并且坚持做的事情吗}
有必要一定要在学习之外找到一件喜欢并且坚持做的事情吗(包括acm竞赛,健身),不然大学只学习会不会看起来很没意义?
\begin{itemize}
\item {\name 匿名学长学姐}  \quad 不后悔就好。
\item {\name 12 陈思奇}\quad 大学的意义,在于找到你喜欢的方向,明白自己今后想成为什么样的人。而不是好好学习天天向上。
\item {\name 12 陈愉}  \quad 看自己希望自己成为一个什么样的人
\item  {\name 12 田博雨}  \quad 只学习是很没意义的,20多岁的年纪,是该花些时间享受青春的。多锻炼,搞搞体育运动。
\item {\name 03 曲文涛}\quad 参加acm竞赛是很有意义的事情,非常锻炼人,有机会参加的话一定要珍惜机会。
\item  {\name 05 刘畅}  \quad 我业余喜欢打dota,看跑男跟我是歌手,但是赶paper的时候(一年大概有300天以上在赶paper)。我一般不太上课。我本科的时候前两年在弄acm竞赛,后两年在做research。到graduate school之后基本是不去上课的,只花最少的时间拿到A就可以了。我的大部分时间都在做research。以我目前的阶段来说,除了我自己的research,其他的大部分事对我来说看起来很没有意义。
\item  {\name 03 刘渊杰}  \quad 有的 生活应该丰富多彩一下 research很重要 但是也可以去尝试些别的东西 我们班就有毕业后完全不做research并且也很成功的例子
\item  {\name 02 杨林骥}  \quad 要 人生又不是读书 不然长大会后悔
\item  {\name 11 杨宽}  \quad 有必要找一件非常喜欢的事,但不一定是学习以外。如果对学习非常有兴趣其他事情都不做也不要紧。
\item  {\name 09 朱虹宇}  \quad 一切的一切,让自己保持愉悦。如果你能一直学术还能一直开心那也成。你要是一直玩而不学术你自己应该也要紧张。如果你为了学术而让自己平时过的很烦闷可能你学术也搞不好。
\item  {\name 07 刘斌}  \quad 个人的理解是高中是训练课业基础的黄金期,大学是树立人生观价值观的黄金期。感觉四年下来,变化最大的是人的想法,思维方式,待人接物等等。不必拘泥具体的事情,只要能和朋友在一起,大学就很有意义。
\item  {\name 12 谢其哲}  \quad 坚持睡觉和锻炼,以健康为代价省下来的时间都是要还的。而且人的精力水平、专注能力是由健康影响的,也会影响学习工作。
\end{itemize}

\begin{comment}
\subsection{如何处理生活和工作的平衡的?}
包括恋爱:如何均衡恋爱与学习?恋爱会影响学习吗?面对异地恋要怎么办?
\begin{itemize}
\item {\name 12 陈思奇}\quad  1.分手;2.坚持很久以后再分手;3.把女友接到异国;4.柏拉图式恋爱
\item  {\name 05 刘畅}  \quad 恋爱两个人会有一段时间是想要腻在一起的,这段时间可能会影响。如果两个人到了平淡期,意识到两个人都有工作要做就好了。如果可以一个月见面4次的话我觉得还好。否则就需要考虑一下怎么改变现状.
\item  {\name 02 杨林骥}  \quad 不谈 影响 但是会后悔
\item  {\name 12 谢其哲}  \quad 没恋爱的时候主要跟同学朋友玩,恋爱了以后就要陪女朋友玩了,人不可能不玩,所以恋爱可以不影响学习。如果两个人能忍受异地恋能耐得住寂寞,其实并不会导致分手,两个人对于未来的规划不一致才会,比如如果一个人一定要呆在国内,另一个人想在国外,不确定什么时候回国,就很可能分手。
\end{itemize}
\end{comment}



\section{经验分享}
\addtocounter{section}{1}
\setcounter{subsection}{0}

\subsection{{\name 12 谢其哲}}

【关键字】主动,Passion,self-motivated

记得我问晓敏老师他对于数学的看法时,他说:“I think it is about passion, about what we really like in life, about taste, beauty, meaning, and very importantly, luck.” 在大学期间,很幸运遇到了这些热爱数学、计算机科学、艺术和改变世界的人。他们让我看到了另外一种生活,一种以创造而不是享受为乐的生活。即使这样的生活有时是痛苦的,对于创造和美的追求有时就像魔鬼盘踞上心头,让人一刻也不能宁静,他们也依然沉醉其中。

虽然我比较愚笨,受应试教育的影响颇深,但我很幸运的进入了ACM班,俞凯老师的语音实验室和微软亚洲研究院的机器学习组。有优秀老师和同学的陪伴,耳濡目染之下也在对话系统和自然语言处理领域的科研中找到了这种美。回望接触科研之前,习惯应试教育的我计较着分数而从不知道自己想要什么,更不知道靠自己的努力去主动追寻。

有一句话是“不要用战术上的勤奋掩盖战略上的懒惰”。战术上的勤奋让我在大一大二中努力学习,我能在一场场考试中取得高分,然而每次考完试总觉得精疲力竭,精力透支的疲惫超过了获取知识的喜悦。战略上的懒惰也体现在把大部分时间不加思考的花在学习课程上,疲惫的我没时间去阅读前人智慧的书籍,没精力参与科研项目。当然把时间完全投入到科研,不顾课程也是不可取的,课程上学习的内容很重要,有很多是研究的基石,而老师的指导能让我们少走很多弯路。只是没有好奇心的学习对于我来说是非常低效的。而我如果早点进行科研和系统的阅读,则会让我更早的发现自己的兴趣,更早的发现原来这些知识是有用的,也就能让我更好的学习。

其实战略的懒惰是一种习惯性的被动,被动的接受生活给我们的位置和任务,而主动的人往往拥有更多机会。主动包含很多方面,最重要的是主动的找到自己热爱的东西,也包括主动的参与重要的项目,主动的表达见解,主动的展示能力,或者在申请中主动的联系老师等等。懒惰是人的天性,而主动是带有侵略性的争取,但往往伴随着走出comfort zone的挫败和痛苦。

我们常觉得主动联系老师会让老师很烦,但其实他们比较喜欢主动的学生,认为主动的学生能很好的承担科研的责任。主动的展示自己也很重要,比如在国外的IT公司,华人做到高层的比印度人少很多。我相信其中一个原因是我们的英语不好,也不善于展示自己,即使做出来很好的成果也很难让人知晓。主动的参与重要的项目和发表见解也是锻炼自己能力的机会。我在微软亚洲研究院的时候,有一次导师提到了一个有意思的想法,于是我除了做完规定的项目,也把这个想法调研、实现了一遍,导师因此对我印象很深刻,我也有幸得到了他更多的指导和信任。以前的我担心自己的研究实力太弱因此不敢在组会上提问,而在微软亚洲研究院的组会上我也经常主动参与讨论和建议,有了一定的研究积累之后不仅能提出好问题,也能提出建设性建议,在与大牛的讨论中锻炼了自己的批判性思维。

John Hopcroft教授和很多优秀的教授们都曾不断告诉我们Find your love。确实人只有在从事自己热爱的事业时才会有一种使命感促使他不分昼夜的专注,最终有所成就。但关于怎么找到自己热爱的领域,其实并没有什么人告诉我们,我们必须自己带着批判思维去寻找。甚至我们永远也不能确信自己找到了,因为我们的想法也会随着环境的变迁而改变。我受Y Combinator创始人Paul Graham的博客和Cal Newport教授的书启发较多。有很多人说应该follow your passion,但其实我们对大多数职业并没有清晰的看法,所以并不知道自己的passion,我们在做一件事前很难知道自己是否喜欢,而且我们的passion受到很多因素的影响,在选择研究领域时,我们需要考虑该领域内是否还有发展机会,大家都在做incremental work还是有big leaps,以及这个研究领域的问题大家是否关心,是一个系统中很无关紧要的一部分还是很重要的部分。比如可能你觉得某个领域的研究很重要,但进入这个领域以后发现这个领域已经很多年没有本质性进步,只是对每种数据集改变一下特征,套用以前的方法,可能一开始的热情也会慢慢消磨殆尽。同样,在选择一份职业的时候,除了对于工作内容的感兴趣程序,可能还要考虑自由度怎么样,比如我曾问过一位做过consulting的研究员两份工作的区别,他说consulting行业的工作时间是由客户决定的,周日只要客户一个电话,就要马上去上班;做科研可能更辛苦一点,但时间支配比较自由。

我们怎么发现自己是否喜欢一个领域或职业呢,我读了Paul Graham的《What you'll wish you'd known》后认为比较好的方法是work hard on hard and interesting projects。如果过一段时间发现不喜欢就换一个。在项目中尽全力才能保证自己不是被难度吓倒了。因为人总有惰性,总是会抵触自己不熟悉的东西,面对一个新领域,我们可能会因为畏惧而感到这个领域的工作没有意思或者不适合自己。但其实就像锻炼后的肌肉酸痛是肌肉生长的标志,这种抵触感也是我们在获取新技能的象征,而不是浪费时间在已经获得的技能上。但就像普通人不可能一周锻炼成施瓦辛格,我们也不可能一个月成为一个领域的大牛,要进行长期的积累。我记得我跟孙锴一起做DSTC-4比赛的时候发现自己的分析问题能力远不如他,所以一直很苦恼是不是自己天生的能力不足,后来发现他很早就开始写五子棋AI,并且在这个任务上打败了世界上其他所有的选手,我才相信自己与他的差距主要来自于没有长时间的积累而不是天赋的缺失。

我们可以通过竞赛项目或者开源项目找到hard and interesting projects。比如Kaggle上企业悬赏的比赛,或者找Linux内核的bug,或者五子棋等等。或者是找到自己周围最强的人,看看他们在做什么,除了周围的同学也包括领域内优秀的同龄研究者。我发现大学中大部分优秀的人都分布在优秀的实验室中,我觉得原因可能是优秀的教授会吸引优秀的学生,进一步的实验室的优秀人才变多后,会有一种追求卓越的气氛,实验室的工作也就更加吸引人。我觉得如果觉得自己没有毅力一个人做self-motivated projects的话,一个比较好的选择是尽早进入实验室。但是需要注意的是如果能力不够,导师很有可能分配一些无意义的重复性工作,比如做码农,标数据等。我觉得可以常常反思自己是不是在获得别人没有的,有价值的技能,如果不是的话尽早与导师沟通,表示自己有能力做一些更有挑战性的事,然后拿出结果来证明自己的能力。

我父母曾经常说的一句话是:“考上大学就轻松了。”事实上,我会觉得自己的知识永远都不够,生活永远都不会有轻松之时,然而得益与在ACM班的成长,我非常喜欢这种不轻松的生活。最后引用王安石的一句话,世之奇伟、瑰怪,非常之观,常在于险远,而人之所罕至焉,故非有志者不能至也。尽吾志也而不能至者,可以无悔矣。

\subsection{{\name 12 陈皓}}

【关键词】:健身、托福口语

健身:

为什么要健身?可以是提高体能,可以是为了外形更好看,也可以是为了研究/学习之余的放松。开始的理由可以有很多,好处也是显而易见的。再说,身为计算机系的学生,平时长期面对电脑工作,如果没有运动的辅助,很容易落下颈椎等部位的疾病,身形臃肿也是在所难免。所以我觉得,如果有条件,大家最好都能养成健身习惯。然而健身见效是一个长期的过程,所以大多数人都没有坚持下来(在学校健身房看到各种新面孔来了又走,老面孔只有那么几个)。

以我自己的经验,坚持下来需要如下几点:首先,连续性。放弃健身常见的模式是“今天好忙啊不去了,咦这几天都好忙,算了吧”。但学校健身房其实非常方便,一次有效率的训练也并不需要拖得很长,而且有经验的话还可以在寝室进行自重训练,这么点时间还是能挤出来的。以我自己为例,我在大二准备GRE考试+面临编译器和数理逻辑等等多重挑战的情况下(大概是我四年最忙的时候),依然保持了几乎三天两练的强度。保持连续性是需要一定的决心的,尤其是在变忙以及环境变化(比如去外面交流实习之类)的情况下,需要主动来计划调整。

其次是效率。之前提到,一次有效率的训练并不需要很久。其实如果训练强度跟上,即便是运动员也撑不了很长时间。那些动辄一两个小时的训练往往都是把健身房当作社交场所,时间都花在了闲聊和刷手机上。不要这样做。用秒表控制训练组间间歇时间,保持训练强度,对新手来说,训练加上后续放松时间不过每天一小时左右。

最后,享受过程。它可以让你从一天的精神疲劳中走出来,何况坚持下来,你的身体也会慢慢产生变化。坚持个半年、一年时间,你会发现自己脱胎换骨。所以为什么不享受这个过程呢?

托福口语:

本人托福分数不高,但口语还是拿到了不错的分数(24),全班应该排到第二了,所以感觉还是有资格一说的。

如果离考试时间还很长($geq$1年?),那准备时间应该是足够了。只差半个月的我也不知道该怎么提高,据说我们班也有短时间快速提升的案例。

时间很长的话,就可以提高一项短时间没法迅速掌握的能力了:语音语调。虽然据说这不是托福考试的重点,但能把话说溜了还是能给自己带来不少自信,最起码语速快了不用担心话说不完...... 练习的方法就是模仿native speaker说话。我联系的材料是VOA standard,一句一句模仿播音员说的话,并且同步录下来,每句话录到满意为止。这样练习时间长了,我会发现自己对连读等语音现象非常熟悉,而且语调也很自然。有兴趣的同学可以练习一些简单的英文rap,我练过Linkin Park的In the End还有Faint等等,感觉也是挺有帮助的。

此外我还有平常没事用英文来说一段的习惯。题材不限,基本是想到什么来什么。不过这个仅限于没人的时候,不然扰民...... 

然后就是考前,可以正常准备了。对题材准备充分一点,加上平时打好的基础,24相信不是太难的。

\subsection{{\name 12 赵卓越}}

【关键词】 大作业、课程、实验室

我觉得编译器、操作系统和数据库大作业是ACM班最有趣的课程,对于我们的工程能力和代码能力都很有帮助,从中可以学习到很多刷oj和写作业不能学习到的编程技巧,在今后的研究中会很有用。在做大作业时,我不建议去参考过去学长或者同学的代码,但是可以互相讨论想要实现的功能和算法,阅读的论文和书籍,并根据自己的能力确定大作业要做到什么程度。

ACM班的课程比较紧张,一学期中会有多门重要的专业课和大作业,以致于一直都时间很赶。我觉得对于课程中必须做的事情,比如作业、小project,不要赶在deadline的时候做,不然很可能几件事情都赶在一起做了,以致于要通宵赶工。早点做完的话,就可以有计划地安排时间,让自己的学习生活不会太紧张。通识课之类的课程如果可以的话,尽量早一些修完,不然越往后会越没有时间上。

实验室实践很重要,在做研究的时候,如果没有学长带的话,最好主动地去学习相关的东西,并且多动手写代码,不然每次开会时就会没有什么可以与导师讨论的,学习不到研究需要的技能。

\subsection{{\name 12 彭燕庆}}

【关键词】:申请 托福 学生活动
只想说一句,学校网站上写托福No minimum requirement并不一定真的是没有要求啊啊啊\\
无论如何也要把托福刷到至少100,口语至少22,不要向我这样99就心存侥幸\\
只因为1分的T没有去成dream school好怨念……department都要求grad school发offer了被托福不到100为理由驳回了orz\\
总分100几乎没有任何问题了,稳妥一点可以105。口语22是底线,有能力的话能考多高就考多高吧。不过其实口语分数并不代表英语交流水平……面试时和教授谈笑风生最后口语还是一团糟orz

身为曾经的宣传部长/辩论队教练/团支书\\
语重心长地和学弟学妹们说一句\\
各种课余活动绝对不要超过一项。选择自己最喜爱的就好,其余的都推掉.\\
不然绝对会累死的。也会影响成绩。\\
当然……如果你把多下来的时间用在打游戏上的话……还不如去找点正经事情干orz

% \makeresources	% Build "Related Resources"
\end{document}
